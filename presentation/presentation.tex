\documentclass{beamer}
\usepackage[T1]{fontenc}
\usepackage[utf8]{inputenc}
\usepackage{lmodern}
\usepackage[polish]{babel}
\usepackage{graphicx}

\usetheme{AGH}

\title[Interpolacja obrazu barwnego]{Równoległa interpolacja obrazu barwnego
  z~kamery cyfrowej}

\author[B. Bułat, T. Drzewiecki]{Bartłomiej Bułat, Tomasz Drzewiecki}

\date[2011]{24.01.2012}

\institute[AGH]
{Wydział EAIiIB\\ 
Katedra Automatyki i Inżynierii Biomedycznej
}

\setbeamertemplate{itemize item}{$\maltese$}

\begin{document}

{
\usebackgroundtemplate{\includegraphics[width=\paperwidth]{titlepagepl}} % wersja polska
 \begin{frame}
   \titlepage
 \end{frame}
}

%---------------------------------------------------------------------------

%% Figure example
%\begin{figure}
%  \includegraphics[width=0.5\textwidth]{filename}
%  \caption{Figure caption}
%  \label{fig:label}
%\end{figure}

\begin{frame}
\frametitle{Wstęp}

\end{frame}

\begin{frame}
  \frametitle{Implementacja kontrollera po stronie CPU}
  W~celu łatwiejszej, szybszej oraz generującej mniej błędów implementacji stworzono maly framework klas realizujących obsługę OpenCL od strony procesora.
Schemat jego struktury jest przedstawiony na rys. \ref{fig:class_diagram} na następnym slajdzie.
\end{frame}

\begin{frame}
  \frametitle{Diagram klas}
\begin{figure}
  \centering
  \includegraphics[width=0.6\linewidth]{class_diagram}
  \caption{Diagram klas stworzonego frameworku.}
  \label{fig:class_diagram}
\end{figure}
  
\end{frame}

\begin{frame}
  \frametitle{Testowanie - sprzęt}
  Testy algorytmu zostały zrealizowane z użyciem dwóch kart graficznych o~parametrach:
\begin{center}
   \begin{tabular}{ |l | l | l | }
     \hline
       & GT 555M & GT9800 \\ \hline
     Liczba rdzeni & 144 & 128 \\ \hline
     Częstotliwość rdzenia & 1250MHz & 600MHz \\ \hline
     Częstotliwość pamięci & 1800MHz & 900MHz \\ \hline
     Magistrala pamięci & 128bit & 256bit \\ \hline
   \end{tabular}
\end{center}
\end{frame}
\begin{frame}
  \frametitle{Testowanie - procedura}
  Algorytmy testowano na 79 obrazach o wymiarach 2546x2058 pikseli. Do każdego testu użyto tych samych obrazów, które były zapisane na dysku. Czas odczytywania obrazów z plików nie był doliczany do czasu obliczeń. Liczono czas wykonania wszystkich operacji koniecznych do wykonania algorytmu (np. kopiowanie do pamięci karty graficznej) oraz osobno czas wykonania kerneli.

Jako czas referencyjny został użyty czas wykonania obliczeń implementacji algorytmu w bibliotece OpenCV wykonanej na procesorze.
\end{frame}
\begin{frame}
  \frametitle{Podsumowanie}
  
\end{frame}

\end{document}

